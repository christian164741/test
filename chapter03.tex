
\chapter{Algebra und Gleichungen}
\label{chap:III_algebra}
\label{chap:III_qubit}
\setcounter{section}{3}
\setcounter{subsection}{0}
\setcounter{subsubsection}{1}
\setcounter{secnumdepth}{3}


\subsection{Die arabische Mathematik und die Geburt der Algebra}
Die arabische Mathematik entwickelte Methoden, die weit über das Rechnen hinausgingen. 
Mit der systematischen Behandlung von Gleichungen entstand die Algebra als eigenes Gebiet. 

\subsection{Gleichungen als Werkzeuge zur Problemlösung}
Gleichungen wurden zum universellen Werkzeug, um unbekannte Größen zu berechnen. 
Von einfachen linearen Gleichungen bis zu quadratischen Formen zeigte sich ihre enorme Nützlichkeit. 

\subsection{Die Entstehung der komplexen Zahlen}
Bei der Lösung höherer Gleichungen traten scheinbar „unmögliche“ Zahlen auf. 
Aus dieser Notwendigkeit heraus entstanden die komplexen Zahlen, die bald eine tiefe Bedeutung erhielten. 

\subsection{Mathematik als System von Regeln}
Mit Algebra und Gleichungen begann die Mathematik, sich stärker als ein in sich geschlossenes Regelwerk zu verstehen. 
Diese Sichtweise prägte die weitere Entwicklung bis in die Moderne. 

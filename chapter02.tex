\chapter{Zahlen und Wirklichkeit}
\label{chap:II_zahlen}
\label{chap:II_grundlagen}
\setcounter{section}{2}
\setcounter{subsection}{0}
\setcounter{subsubsection}{1}
\setcounter{secnumdepth}{3}
% Boxen-Stile definieren
\subsection{Natürliche Zahlen und Zählen}
\label{sec:2.1_natuerliche_zahlen}

Die Geschichte der \emph{Zahlen}\index{Zahlen} beginnt mit dem Bedürfnis des Menschen, 
seine Umwelt zu ordnen. 
Lange bevor es \emph{Schrift}\index{Schrift} oder 
\emph{Rechenverfahren}\index{Rechenverfahren} gab, 
nutzte man das \emph{Zählen}\index{Zählen}, um Tiere, Werkzeuge oder Tage festzuhalten. 
Das Zählen war eine grundlegende \emph{Kulturtechnik}\index{Kulturtechnik}, 
ohne die \emph{Handel}\index{Handel}, 
\emph{Planung}\index{Planung} und 
\emph{Wissenschaft}\index{Wissenschaft} 
nicht denkbar gewesen wären.

\subsubsection*{Früheste Ansätze}
Archäologische Funde\index{Archäologie} deuten darauf hin, dass bereits vor 
zehntausenden von Jahren \emph{Zählhilfen}\index{Zählhilfen} benutzt wurden. 
Besonders bekannt ist der folgende Fund: Der Ishango-Knochen.
\newpage
\noindent
\HistoryBox{Der Ishango-Knochen}{box:ishango}{%
	Der \emph{Ishango-Knochen}\index{Ishango-Knochen} wurde 1960 von 
	Jean de Heinzelin de Braucourt\index{Heinzelin de Braucourt, Jean de} 
	in Zentralafrika entdeckt und ist etwa 20.000 Jahre alt. 
	
	Auf dem Knochen finden sich drei Spalten von Einkerbungen, 
	die nicht zufällig verteilt sind, sondern in Gruppen geordnet erscheinen. 
	Wahrscheinlich diente er als eine Art \emph{Zählstab}\index{Zählstab} – 
	möglicherweise zur Beobachtung des \emph{Mondzyklus}\index{Mondzyklus} 
	oder zur Zählung von Vorräten.
	
	Besonders auffällig: Einige Gruppierungen entsprechen \emph{Primzahlen}\index{Primzahlen} 
	(11, 13, 17, 19). Ob dies Zufall oder Absicht war, bleibt offen. 
	Der Knochen zeigt jedoch, dass Menschen schon in der \emph{Altsteinzeit}\index{Altsteinzeit} 
	Zahlen nutzten, um Muster in ihrer Welt zu erkennen und festzuhalten.
}

\begin{figure}[ht]
	\centering
	\includegraphics[width=0.55\textwidth]{bilder/Ishango_bone.jpg}
	\caption{Der Ishango-Knochen, ein etwa 20.000 Jahre altes Artefakt mit Einkerbungen, 
		entdeckt von Jean de Heinzelin de Braucourt. 
		Heute im Königlichen Institut für Naturwissenschaften in Brüssel ausgestellt. 
		Foto: Joeykentin, Lizenz CC BY-SA 4.0, via Wikimedia Commons.}
	\label{fig:ishango}
\end{figure}

\HistoryBox{Weitere Zählhilfen: Kerbhölzer}{box:kerbhoelzer}{%
	Nicht nur in Afrika, auch in Europa oder Asien nutzten Menschen Kerbhölzer\index{Kerbhölzer}, 
	um Mengen festzuhalten. 
	Ein berühmtes Beispiel ist das sogenannte \emph{Lebombo-Knochenstück}\index{Lebombo-Knochen}, 
	etwa 35.000 Jahre alt, das in Südafrika gefunden wurde. 
	Auch im europäischen Alpenraum wurden Kerbhölzer bis in die Neuzeit verwendet, 
	etwa zur Dokumentation von Schulden oder Abgaben. 
	
	Diese Funde zeigen: Überall griff man unabhängig zur gleichen Idee – 
	Einkerbungen als elementare Zählhilfe.
}

\HistoryBox{Fundort und Aufbewahrung}{box:ishango_fundort}{%
	Der Ishango-Knochen wurde in den 1950er-Jahren am Ufer des Edwardsees in der heutigen 
	Demokratischen Republik Kongo gefunden. 
	Heute wird das Artefakt im \emph{Königlichen Institut für Naturwissenschaften in Brüssel}
	aufbewahrt und ausgestellt. 
	Damit gehört er zu den bedeutendsten Zeugnissen früher Zahlverwendung.
}

\subsubsection*{Vom Zählen zum Zahlbegriff}
Neben solchen archäologischen Funden zeigt auch die Sprachforschung, 
wie Menschen ihre Umwelt zunächst nur grob in kleine und große Mengen unterteilten. 
Anfangs unterschied man lediglich: \glqq eins, zwei, viele\grqq. 
Bei einigen heute noch existierenden Kulturen\index{Kulturen} 
finden wir dieses eingeschränkte Zahlverständnis wieder. 

\DidaktikBox{„Eins, zwei, viele“}{box:eins_zwei_viele}{%
	In einigen traditionellen Kulturen gibt es nur sehr wenige Zahlwörter\index{Zahlwörter}. 
	Oft unterscheidet die Sprache lediglich zwischen \glqq eins\grqq, \glqq zwei\grqq 
	und \glqq viele\grqq. 
	
	Ein Beispiel sind bestimmte Völker in Papua-Neuguinea\index{Papua-Neuguinea} 
	oder in Australien\index{Australien}, deren \emph{Zahlensystem}\index{Zahlensystem} so begrenzt ist. 
	Für Mengen über zwei wird einfach das Wort für \glqq viele\grqq verwendet. 
	
	Dies zeigt: \emph{Zahlen}\index{Zahlen!kulturelle Entwicklung} sind keine 
	selbstverständliche Gegebenheit, sondern eine kulturelle Errungenschaft. 
	Das Zählen entwickelt sich erst dann weiter, wenn die Gesellschaft 
	konkrete Bedürfnisse wie \emph{Handel}\index{Handel}, 
	\emph{Vorratswirtschaft}\index{Vorratswirtschaft} 
	oder \emph{Kalenderrechnung}\index{Kalenderrechnung} hat.
}

\HinweisBox{Zählen als universale Struktur}{box:zaehlen_universell}{%
	Ob in Afrika, Asien, Europa oder Ozeanien – überall entwickelten Menschen 
	Hilfsmittel und Begriffe zum Zählen. 
	Dies zeigt: Das Bedürfnis, die Welt in Zahlen zu fassen, ist universell 
	und spiegelt eine grundlegende Struktur der Wirklichkeit wider.
}

\subsection{Irrationale Zahlen – die Krise der Griechen}
\label{sec:2.2_irrat}

\subsubsection{Ausgangspunkt: Pythagoras und die Pythagoreer}
Die frühen griechischen Mathematiker\index{Griechen}, vor allem die Schule der 
\emph{Pythagoreer}\index{Pythagoreer}, waren überzeugt, dass sich die Welt vollständig 
durch \emph{ganze Zahlen}\index{ganze Zahlen} und \emph{Brüche}\index{Brüche} beschreiben lässt. 
Ihr Leitsatz lautete: \glqq Alles ist Zahl.\grqq 
Der Satz des Pythagoras\index{Satz des Pythagoras} war dafür das zentrale Symbol.

\subsubsection{Erste Erkenntnisse: Inkommensurabilität}
Beim Quadrat mit Seitenlänge $1$ wollten die Pythagoreer\index{Pythagoreer} 
die Länge der Diagonale berechnen. 
Nach dem Satz des Pythagoras\index{Satz des Pythagoras} gilt 
$a^2+b^2=c^2$. Für ein Quadrat mit $a=b=1$ folgt daraus:
\[
c^2 = 1^2 + 1^2 = 2 \quad \Rightarrow \quad c = \sqrt{2}.
\]

Die Zahl $\sqrt{2}$ ließ sich jedoch nicht als Bruch zweier ganzer Zahlen darstellen. 
Damit war klar: Es gibt \emph{inkommensurable Längen}\index{inkommensurabel}, 
die kein gemeinsames Maß besitzen. 
Die Seitenlänge $1$ und die Diagonale $\sqrt{2}$ sind inkommensurabel.

Diese Erkenntnis war für die Pythagoreer\index{Pythagoreer} ein Schock, 
denn sie widerlegte ihre Überzeugung, dass die Welt ausschließlich 
durch ganze Zahlen\index{ganze Zahlen} und Brüche\index{Brüche} erklärbar sei.


\subsubsection{Die Krise der Griechen}
Für die Pythagoreer war diese Entdeckung eine Katastrophe. 
Ihre Weltordnung, die auf ganzen Zahlen und Brüchen beruhte, geriet ins Wanken. 

\HistoryBox{Die Legende von Hippasos}{box:hippasos}{%
	Der Pythagoreer Hippasos\index{Hippasos} soll – der Überlieferung nach – 
	die Irrationalität der Quadratwurzel aus $2$ entdeckt haben. 
	Die Legende berichtet, Hippasos sei von seinen Brüdern im Meer ertränkt worden, 
	weil er dieses Geheimnis preisgegeben hatte. 
	Ob dies historisch zutrifft, ist unklar – sicher ist jedoch, 
	dass die Entdeckung der Irrationalität die griechische Mathematik in eine tiefe Krise stürzte.
}

\subsubsection{Der Beweis}
Einen strengen Beweis für die Irrationalität lieferte erst später 
Euklid\index{Euklid} in seinen \emph{Elementen}\index{Elemente (Euklid)}. 
Dort entwickelte er eine Theorie der inkommensurablen Größen. 

\MatheBox{Widerspruchsbeweis für $\sqrt{2}$}{box:beweis_wurzel2}{%
	Angenommen, $\sqrt{2}$ sei rational, also $\sqrt{2}=\tfrac{p}{q}$ mit 
	ganzen Zahlen $p,q$, die teilerfremd sind. 
	Dann gilt $2q^2=p^2$. 
	Daraus folgt: $p^2$ ist gerade, also ist auch $p$ gerade. 
	Schreiben wir $p=2r$, dann gilt $2q^2=(2r)^2=4r^2$, also $q^2=2r^2$. 
	Damit ist auch $q$ gerade. 
	
	Somit sind $p$ und $q$ beide gerade, also nicht teilerfremd. 
	Dies widerspricht der Annahme. 
	Damit ist $\sqrt{2}$ irrational.
}

\DidaktikBox{Warum $\sqrt{2}$ keine Bruchzahl sein kann}{box:idee_wurzel2}{%
	Stellen wir uns vor, man könnte die Zahl $\sqrt{2}$ als Bruch $\tfrac{p}{q}$ schreiben, 
	also als Verhältnis zweier ganzer Zahlen. 
	Dann müsste es ein einfaches Maß geben, mit dem man sowohl die Seite als auch 
	die Diagonale eines Quadrats abmessen könnte. 
	
	Doch genau das klappt nicht: 
	Ganz gleich, welchen Bruch man probiert, die Diagonale passt nie in ein ganzes Verhältnis zur Seite. 
	Man stößt immer auf Widersprüche – entweder bleiben Reste übrig oder die Zahlen müssten gleichzeitig 
	gerade und ungerade sein. 
	
	Die Botschaft ist einfach: 
	$\sqrt{2}$ gehört nicht zur Welt der Brüche, sondern eröffnet eine neue Zahlenwelt – 
	die der irrationalen Zahlen.
}

\subsection*{Zahlen als Grundlage der Kultur}
Mit dem Zählen entstand ein neues Denken: Viehherden\index{Viehherden} konnten verwaltet, 
Vorräte\index{Vorräte} geplant, Handel betrieben und Kalender\index{Kalender} entwickelt werden. 
Damit legten die \emph{natürlichen Zahlen}\index{natürliche Zahlen} den Grundstein 
für jede spätere Mathematik\index{Mathematik} und für das Aufblühen von Hochkulturen\index{Hochkulturen}.


\subsection{Null und negative Zahlen}
Die Einführung der Null und der negativen Zahlen war keine Selbstverständlichkeit. 
Sie öffnete jedoch den Weg zu neuen Rechenmethoden. 

\subsection{Komplexe Zahlen – eine unerwartete Erweiterung}
Die komplexen Zahlen begannen als Hilfskonstruktion zur Lösung von Gleichungen. 
Heute sind sie unverzichtbar in Physik und Technik. 
